% !TEX encoding = UTF-8 Unicode
%!TEX root = thesis.tex
% !TEX spellcheck = en-US
%%=========================================

\chapter{Setup guide}
\section*{Initial requirements}
\begin{itemize}
    \item A server running Debian (Jessie or newer) with more than 2GB of RAM and 10GB of free space. The server should also have git installed.
    \item A domain pointing to the server mentioned above.
    \item SSL/TLS certificates for the domain mentioned above SSL/TLS is not required, but it is recommend as you would otherwise have to change some of the services.
    \footnote{Certificates can be obtained for free by following: \url{https://www.digitalocean.com/community/tutorials/how-to-secure-nginx-with-let-s-encrypt-on-ubuntu-14-04}}
    \item Prior experience with Linux and the command line.
\end{itemize}

\section*{Step 1 - Installing the required software}
Begin by installing docker, docker-compose and nginx, which is required to run the system..
If the server is running Debian Jessie, you must first add the Debian Stretch package repository, as Nginx 1.10 or newer is required.
\clearpage % As the command below would otherwise be split between two pages and a footnote(!).
The Stretch package repository can be used by adding:
\begin{verbatim}
    deb http://http.debian.org/debian stretch main
\end{verbatim}
to /etc/apt/sources.list, and then running:
\begin{verbatim}
    apt-get update
\end{verbatim}
in order to obtain the packages from the newly added repository.
Nginx is then installed by running:
\begin{verbatim}
    apt-get -t stretch install nginx-extras
\end{verbatim}
or just
\begin{verbatim}
    apt-get install nginx-extras
\end{verbatim}
if you are running a newer version than Jessie.

After having installed Nginx, follow:
\begin{itemize}
    \item \url{https://docs.docker.com/engine/installation/linux/debian/}
    \item \url{https://docs.docker.com/compose/install/}
    \item \url{https://github.com/jwilder/docker-gen}
\end{itemize}
to install docker, docker-compose and docker-gen.


\section*{Step 2 - Adding configuration files}
Begin by running:
\begin{verbatim}
    git clone https://github.com/microserv/deploy
\end{verbatim}
to obtain the configuration files needed to deploy the system.

Now enter the newly created \textit{deploy} folder, and replace all references to \textit{despina.128.no} with your own domain name.
This can also be done by executing the following command:
\begin{verbatim}
    find . -type f -exec sed -i 's/despina.128.no/<YOUR DOMAIN HERE>/g' {} +
\end{verbatim}
Then, copy the docker-gen configuration file \textit{deploy/docker-gen/docker-gen.cfg} to \textit{/etc/} and the docker-gen systemd unit file \textit{deploy/docker-gen/docker-gen.service} to \textit{/etc/systemd/system}.

After copying the files mentioned above, copy the nginx configuration file \textit{deploy/nginx/nginx.conf} to \textit{/etc/nginx/} and the nginx docker-gen template \textit{deploy/docker-gen/nginx.conf.tmpl} to \textit{/etc/nginx/}

Now enable the docker-gen systemd unit file by running:
\begin{verbatim}
    systemctl daemon-reload
    systemctl start docker-gen
\end{verbatim}

\section*{Step 3 - Launching the system}
The system is launched using docker-compose. Docker-compose automatically fetches the Docker images required to launch the containers. The Docker images is by default fetched by Docker from a private repository hosted at 128.no:8080. To be able to access this repository, you must first log-in. To log-in execute:
\begin{verbatim}
    docker login dockerisfun:QRN5A2So
\end{verbatim}

Now enter the \textit{deploy/docker} folder, and run:
\begin{verbatim}
    docker-compose -d
\end{verbatim}
to launch the system.

The system should be running and accessible through your domain after about 15 seconds. Then, access the administration panel by first accessing \textit{YOUR-DOMAIN-HERE/auth/admin} and logging in using \textit{admin} as the username, and \textit{aStupidDefaultPasswordYouShouldChange} as the password.

After having logged in, go to \textit{Applications}, then \textit{Publishing}, and replace despina.128.no with your own domain in the \textit{redirect URI} field. The system should now work as intended.

Kibana, elasticsearch and logstash, which is used to monitor the logs produced by the Docker containers can be launched by running the script \textit{start-docker-logging.sh} in the \textit{docker} folder. These services can be stopped by running \textit{stop-docker-logging.sh} in the same folder. When these services are up, the logs can by read by accessing \textit{YOURDOMAINNAMEHERE/kibana/}.